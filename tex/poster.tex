%!LW recipe = latexmk (PdfLaTeX)
\documentclass[a1, portrait]{a0poster} % A0 縦向き

% --- 数学・記号系 ---
\usepackage{amsmath, amsthm, mathtools}
\usepackage{amssymb}  %(amsfontsを自動で含む)

% --- 段落設定 ---
\usepackage{multicol}

% --- 日本語・フォント設定 ---
\usepackage[whole]{bxcjkjatype}
\usepackage[T1]{fontenc}
\usepackage[utf8]{inputenc}

% --- 以下 lstlisting の設定 ---
% --- ソースコード表示 ---
\usepackage{listings}
\usepackage{zi4}  % Leanコード表示用, 任意

% 色の設定
\usepackage{color}
\definecolor{keywordcolor}{rgb}{0.7, 0.1, 0.1}   % red
\definecolor{tacticcolor}{rgb}{0.0, 0.1, 0.6}    % blue
\definecolor{commentcolor}{rgb}{0.4, 0.4, 0.4}   % grey
\definecolor{symbolcolor}{rgb}{0.0, 0.1, 0.6}    % blue
\definecolor{sortcolor}{rgb}{0.1, 0.5, 0.1}      % green
\definecolor{attributecolor}{rgb}{0.7, 0.1, 0.1} % red

% --- lstlistingの設定の読み込み ---
\def\lstlanguagefiles{lstlean.tex}

% 大きさ
% --- lstlistingsの設定 ---
% \lstset{
%   basicstyle=\ttfamily\small,
%   frame=single,
%   mathescape=true,
%   breaklines=true,
%   columns=fullflexible,
%   keepspaces=true,
%   xleftmargin=0.2em,
%   xrightmargin=0.2em,
%   tabsize=2,
%   lineskip=-1pt,
%   aboveskip=0.3em,
%   belowskip=0.3em
% }

% --- フォントサイズの定義  ---
% \renewcommand{\tiny}{\fontsize{10}{12}\selectfont}
% \renewcommand{\scriptsize}{\fontsize{12}{15}\selectfont}
% \renewcommand{\footnotesize}{\fontsize{14}{18}\selectfont}
% \renewcommand{\small}{\fontsize{19}{23}\selectfont}
% \renewcommand{\normalsize}{\fontsize{23}{29}\selectfont}
% \renewcommand{\large}{\fontsize{28}{35}\selectfont}  
% \renewcommand{\Large}{\fontsize{34}{42}\selectfont}
% \renewcommand{\LARGE}{\fontsize{40}{50}\selectfont}
% \renewcommand{\huge}{\fontsize{48}{60}\selectfont}
% \renewcommand{\Huge}{\fontsize{58}{72}\selectfont}
% \renewcommand{\veryHuge}{\fontsize{70}{88}\selectfont}
% \renewcommand{\VeryHuge}{\fontsize{84}{105}\selectfont}
% \renewcommand{\VERYHuge}{\fontsize{100}{125}\selectfont}

% set default language
\lstset{
  language=lean,
  basicstyle=\ttfamily\small,
  breaklines=true,
  frame=single
  }
% --- 以上 lstlisting の設定 ---

\theoremstyle{definition} % ここから下の環境は「立体(真っ直ぐ)」になる
\newtheorem{theorem}{定理}[section]
\newtheorem{lemma}{補題}[section]
\newtheorem{prop}{命題}[section]
\newtheorem{definition}{定義}[section]
\newtheorem{corollary}{系}[section]


\title{\(Lean\) による \(GL(2, \mathbb{Q})\) の有限位数の決定}
\author{氏名 : 森下~善~(北海道大学理学部数学科) \quad 指導教員 : 松下~大介}
\date{~}

\begin{document}
  
% 表紙の作成
\maketitle

\begin{multicols}{2}
  
\section{はじめに}


\section{有限位数の決定}

\begin{definition}[有限位数元]~\

  \(G\) を群とする.

  \(g \in G\) が \(G\) の有限位数元であるとは, \(g^n = 1\)(単位元) を満たすような
  正の整数 \(n\) が存在することである.

  また,このような \(n\) のうち, 最小のものを \(g\) の位数という.

\end{definition}

以下, \(M\) を \(GL(2, \mathbb{Q})\) の有限位数元, \(n\) をその位数とする.

\begin{definition}[最小多項式]~\
  
  最高次数係数が1の多項式 \(m_M(x)\) のうち,

  \(m_M(M) = O\) となるもので次数が最小のものを \(M\) の最小多項式という.

\end{definition}

\begin{definition}[円分多項式]~\

  複素数平面における1の原始n乗根 \(\zeta_n^k = e^{2ki\pi/n}\)(\(1 \leq k \leq n\) かつ \(gcd(k, n) = 1\))
  のすべてを解に持つ多項式,
  \[\Phi_n(x) := \prod_{\substack{1 \leq k \leq n\\ gcd(k, n) = 1}} (x - e^{2ki\pi / n})\]

  を第n円分多項式という.

\end{definition}

\begin{definition}[オイラーのトーシェント関数]~\
  
  正の整数 \(n\) について, \(n\) 以下の正の整数のうち, \(n\) と互いに素なものの個数を
  \(\phi(n)\) と書き, これをオイラーのトーシェント関数という. 
  \[\phi(n) := \# \{a \mid 1 \leq a \leq n, gcd(a, n) = 1\}\]

\end{definition}

\begin{lemma}

  円分多項式の次数は, オイラーのトーシェント関数と等しい.

  \begin{lstlisting}
    lemma cyclotomic_deg_eq_totient (n : ℕ) : (Φ n).natDegree = φ n :=
      natDegree_cyclotomic n ℚ
  \end{lstlisting}

\end{lemma}

\begin{lemma}

  \(M\) の最小多項式 \(m_M(x)\) は
  \[ \Phi_1(x), \Phi_2(x), \Phi_3(x), \Phi_4(x), \Phi_6(x), \Phi_1(x)*\Phi_2(x)\] に限る.

\end{lemma}

\begin{theorem}

  \(GL(2, \mathbb{Q})\) の有限位数元が存在するならば、
  その位数は \(1, 2, 3, 4, 6\) に限る.

\end{theorem}

\begin{theorem}
  
  \(n = 1, 2, 3, 4, 6\) それぞれに対して, 位数が \(n\) となるような
  \(GL(2, \mathbb{Q})\) の元Mが存在する.

\end{theorem}

\section{おわりに}

\section{参考文献}

\end{multicols}

\end{document}