%!LW recipe = latexmk (PdfLaTeX)
\documentclass[a0, portrait]{a0poster} % A0 縦向き

% --- 数学・記号系 ---
\usepackage{amsmath, amsthm, mathtools}
\usepackage{amssymb}  %(amsfontsを自動で含む)

% --- 段落設定 ---
\usepackage{multicol}

% --- 日本語・フォント設定 ---
\usepackage[whole]{bxcjkjatype}
\usepackage[T1]{fontenc}
\usepackage[utf8]{inputenc}

% --- 以下 lstlisting の設定 ---
% --- ソースコード表示 ---
\usepackage{listings}
\usepackage{zi4}  % Leanコード表示用, 任意

% 色の設定
\usepackage{color}
\definecolor{keywordcolor}{rgb}{0.7, 0.1, 0.1}   % red
\definecolor{tacticcolor}{rgb}{0.0, 0.1, 0.6}    % blue
\definecolor{commentcolor}{rgb}{0.4, 0.4, 0.4}   % grey
\definecolor{symbolcolor}{rgb}{0.0, 0.1, 0.6}    % blue
\definecolor{sortcolor}{rgb}{0.1, 0.5, 0.1}      % green
\definecolor{attributecolor}{rgb}{0.7, 0.1, 0.1} % red

% --- lstlistingの設定の読み込み ---
\def\lstlanguagefiles{lstlean.tex}

% set default language
\lstset{language=lean}
% --- 以上 lstlisting の設定 ---

\newtheorem{theorem}{定理}[section]
\newtheorem{lemma}{補題}[section]
\newtheorem{prop}{命題}[section]
\newtheorem{definition}{定義}[section]
\newtheorem{corollary}{系}[section]

\title{Lean による GL(2, Q) の有限位数の決定}
\author{北海道大学理学部数学科4年~学生番号 02220077~ 森下~善 \\指導教員 : 松下~大介}
\date{2026 年 1 月 25 日}

\begin{document}

% 表紙の作成
\maketitle

\section{はじめに}

本稿では, 証明支援系Leanを用いて, 有理数体上の2次一般線型群 GL(2, Q) の有限位数を決定していく.\\
具体的には, GL(2, Q) の有限位数元は, 位数 1, 2, 3, 4, 6 のいずれかであり, 各位数に対して標準的な代表元が存在することを示す.
元 \(g\) が有限位数元であるとは, \(g^n = I\) (単位元) となるような正の整数 \(n\) が存在するということである.


\section{主定理}

\begin{definition}[有限位数元]~\

\(G\) を群, \(I\) を \(G\) の単位元とする.

\(G\) の元 \(g\) が有限位数元であるとは, 
ある正の整数 \(n\) に対して \(g^n = I\) となることである.

また, そのような最小の正の整数 \(n\) を \(g\) の位数と呼ぶ.

\end{definition}


\begin{lemma}

\(M\) を GL(2, Q) の有限位数元とする.
このとき, \(M\) の固有多項式の次数は 2 以下.

\begin{lstlisting}
  lemma charpoly_deg_eq_two : (charpoly M.val).natDegree = 2 :=  -- 固有多項式の次数は2以下
    charpoly_natDegree_eq_dim M.val
\end{lstlisting}
\end{lemma}


\begin{lemma}

\(M\) を GL(2, Q) の有限位数元とする.
このとき, \(M\) の最小多項式は固有多項式を割り切る.

\begin{lstlisting}
lemma minpoly_deg_le_two : (minpoly ℚ M.val).natDegree ≤ 2 :=  -- 最小多項式の次数は2以下
  calc
    _ ≤ (charpoly M.val).natDegree := by
      apply natDegree_le_of_dvd
      · apply minpoly_dvd_charpoly M.val
      · apply ne_zero_of_natDegree_gt
        show 1 < (charpoly M.val).natDegree
        rw [charpoly_deg_eq_two]
        norm_num
    _ = 2 := by rw [charpoly_deg_eq_two]
\end{lstlisting}
\end{lemma}


\begin{lemma}

\(M\) を GL(2, Q) の有限位数元とする.
このとき, \(M\) の最小多項式の次数は 2 以下.

\begin{lstlisting}
lemma minpoly_dvd_charpoly : minpoly ℚ M.val ∣ M.val.charpoly := by -- 最小多項式は固有多項式を割り切る
  apply dvd ℚ M.val
  exact aeval_self_charpoly M.val
\end{lstlisting}
\end{lemma}


\begin{definition}[トーシェント関数]

自然数 \(n\) に対して,

\[
\varphi(n) = \# \{ k \in \mathbb{N} \mid 1 \leq k \leq n, \gcd(k,n) = 1 \}
\]

をトーシェント関数という.

\begin{lstlisting}
def totient (n : ℕ) : ℕ := Finset.card {a ∈ Finset.range n | n.Coprime a}
\end{lstlisting}
\end{definition}


\begin{definition}[円分多項式]
自然数 \(n\) に対して, 円分多項式 \(\Phi_n(x)\) を次で定義する.

\[
\Phi_n(x) = \prod_{\substack{1 \leq k \leq n \\ \gcd(k,n) = 1}} (x - \zeta_n^k)
\]

ただし, \(\zeta_n\) は \(n\) 番目の原始 \(n\) 乗根である.
\end{definition}


\begin{lemma}

円分多項式の次数はトーシェント関数と等しい.

\begin{lstlisting}
lemma cyclotomic_deg_eq_totient (n : ℕ) :
    (cyclotomic n ℚ).natDegree = totient n := by  -- 円分多項式の次数はトーシェント関数
  exact natDegree_cyclotomic n ℚ
\end{lstlisting}
\end{lemma}


\begin{lemma}

\(M\) を GL(2, Q) の有限位数元とする.
このとき, 円分多項式は \(M\) の最小多項式を割り切る.

\end{lemma}


\begin{lemma}

\(n\) を \(M\) の位数とする.
このとき, トーシェント関数 \(\varphi(n)\) は 2 以下である.

\end{lemma}


\begin{lemma}

トーシェント関数 \(\varphi(n)\) が 2 以下となる自然数 \(n\) は, \(n = 1, 2, 3, 4, 6\) のいずれかである.

\end{lemma}


\begin{theorem}

\(GL(2, Q)\) の有限位数元の位数は \(1, 2, 3, 4, 6\) のいずれかである.

\end{theorem}


\begin{theorem}

\(n = 1, 2, 3, 4, 6\) に対して, \(GL(2, Q)\) の有限位数元で位数が \(n\) であるものが存在する.

\end{theorem}


\begin{theorem}

\(GL(2, Q)\) の有限位数元は, 位数 \(1, 2, 3, 4, 6\) のいずれかであり, 各位数に対して標準的な代表元が存在する.

\end{theorem}

\begin{lstlisting}

\end{lstlisting}


\section{おわりに}
本稿では, 証明支援系Leanを用いて, 有理数体上の2次一般線型群 GL(2, Q) の有限位数元の分類を行った.
具体的には, GL(2, Q) の有限位数元は, 位数 1, 2, 3, 4, 6 のいずれかであり, 各位数に対して標準的な代表元が存在することを示した.
今後の課題として, より高次元の一般線型群に対する有限位数元の分類や, 他の体上での類似の問題に取り組むことが挙げられる.

\section{参考文献}

\end{document}