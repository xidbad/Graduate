%!LW recipe = latexmk (PdfLaTeX)
\documentclass[a1, portrait]{a0poster} % A0 縦向き

% --- 数学・記号系 ---
\usepackage{amsmath, amsthm, mathtools}
\usepackage{amssymb}  %(amsfontsを自動で含む)

% --- 段落設定 ---
\usepackage{multicol}

% --- 日本語・フォント設定 ---
\usepackage[whole]{bxcjkjatype}
\usepackage[T1]{fontenc}
\usepackage[utf8]{inputenc}

% --- 以下 lstlisting の設定 ---
% --- ソースコード表示 ---
\usepackage{listings}
\usepackage{zi4}  % Leanコード表示用, 任意

% --- 色の設定 ---
\usepackage{xcolor}
\definecolor{keywordcolor}{rgb}{0.7, 0.1, 0.1}   % red
\definecolor{tacticcolor}{rgb}{0.0, 0.1, 0.6}    % blue
\definecolor{commentcolor}{rgb}{0.4, 0.4, 0.4}   % grey
\definecolor{symbolcolor}{rgb}{0.0, 0.1, 0.6}    % blue
\definecolor{sortcolor}{rgb}{0.1, 0.5, 0.1}      % green
\definecolor{attributecolor}{rgb}{0.7, 0.1, 0.1} % red

% --- lstlistingの設定の読み込み ---
\def\lstlanguagefiles{lstlean.tex}

% set default language
\lstset{
  language=lean,
  basicstyle=\ttfamily\small,
  breaklines=true,
  frame=single,
  columns=fullflexible,
  keepspaces=true,
  % mathescape=true,
  % xleftmargin=0em,
  % xrightmargin=0em,
  % tabsize=2,
  lineskip=-1pt,
  % aboveskip=0.2em,
  belowskip=0em
}
% --- 以上 lstlisting の設定 ---

% --- フォントサイズの定義 (A1ポスター用) ---
\renewcommand{\tiny}{\fontsize{8}{10}\selectfont}
\renewcommand{\scriptsize}{\fontsize{10}{12}\selectfont}
\renewcommand{\footnotesize}{\fontsize{13}{16}\selectfont}
\renewcommand{\small}{\fontsize{16}{20}\selectfont}       % lstlistingなどの基準
\renewcommand{\normalsize}{\fontsize{20}{25}\selectfont}  % 本文の基準
\renewcommand{\large}{\fontsize{24}{30}\selectfont}  
\renewcommand{\Large}{\fontsize{30}{38}\selectfont}       % 見出しなど
\renewcommand{\LARGE}{\fontsize{36}{45}\selectfont}
\renewcommand{\huge}{\fontsize{44}{55}\selectfont}        % セクションタイトルなど
\renewcommand{\Huge}{\fontsize{54}{68}\selectfont}        % ポスタータイトル用
\renewcommand{\veryHuge}{\fontsize{64}{80}\selectfont}
\renewcommand{\VeryHuge}{\fontsize{76}{95}\selectfont}
\renewcommand{\VERYHuge}{\fontsize{90}{112}\selectfont}

\theoremstyle{definition}
\newtheorem{theorem}{定理}[section]
\newtheorem{lemma}{補題}[section]
\newtheorem{prop}{命題}[section]
\newtheorem{definition}{定義}[section]
\newtheorem{corollary}{系}[section]


\title{Leanによる \(GL(2, \mathbb{Q})\) の有限位数の元に関する考察}
\author{氏名 : 森下~善~(北海道大学理学部数学科) \quad 指導教員 : 松下~大介}
\date{~}

% 線の太さを指定(例: 0.4pt が標準的な細線)
\setlength{\columnseprule}{0.4pt}
\setlength{\columnsep}{30pt}

\begin{document}

% 表紙の作成
\maketitle

\begin{multicols}{3}
  
\section{はじめに}

  本稿では, 有理数体上の \(2\) 次一般線型群 \(GL(2, \mathbb{Q})\) における有限位数の元の位数が
  \(1, 2, 3, 4, 6\) に限定されることを証明支援系Leanを用いて形式化していく.
  証明の方針として, まず最小多項式が \(x^n - 1\) を割り切り, \(\mathbb{Q}[X]\) が一意分解整域(UFD)
  であることから, 既約因子が円分多項式に帰着されることを示す.
  そして次数の制約に基づき, 位数を絞り込んでいく.

\section{有限位数の決定}

\begin{definition}[有限位数元]\

  \(G\) を群とする.
  \(g \in G\) が \(G\) の有限位数元であるとは, \(g^n = 1_G\)(単位元) を満たす
  正の整数 \(n\) が存在することである.
  またこのような \(n\) のうち, 最小のものを \(g\) の位数という.

\end{definition}

以下, \(M\) を \(GL(2, \mathbb{Q})\) の有限位数元, \(n\) をその位数とする.


\begin{definition}[最小多項式]\

  最高次数係数が \(1\) である(モニック)多項式 \(m_M(x)\) のうち,
  \(m_M(M) = O\) となるもので次数が最小のものを \(M\) の最小多項式という.

\end{definition}


\begin{lemma}

  最小多項式 \(m_M(x)\) の次数は \(2\) 以下.

  \begin{proof}
    最小多項式が固有多項式を割り切り, 固有多項式の次数が \(2\) であることから示せる.
  \end{proof}

\end{lemma}


\begin{lemma}
  
  最小多項式 \(m_M(x)\) は, \(x^n - 1\) を割り切る.

  \begin{proof}
    \(M^n = I\) より, \((x^n - 1)(M) = O\) が成り立つことと, 最小多項式の定義から示せる.
  \end{proof}

\end{lemma}


\begin{definition}[円分多項式]\

  複素数平面における \(1\) の原始 \(n\) 乗根 \(\zeta_n^k = e^{\frac{2ki\pi}{n}}\)(\(1 \leq k \leq n\) かつ \(gcd(k, n) = 1\) (\(k \in \mathbb{Z}\)))
  のすべてを解に持つ多項式,
  \[\Phi_n(x) := \prod_{\substack{1 \leq k \leq n\\ gcd(k, n) = 1}} (x - \zeta_n^k)\]
  を第 \(n\) 円分多項式という.

\end{definition}


\begin{lemma}

  \(x^n - 1 = \prod_{i \mid n} \Phi_i(x)\).

  \begin{proof}
    \(x^n - 1 = \prod_{1 \leq k \leq n} (x - \zeta_n^k)\) と分解でき, 
    \(i = \frac{n}{gcd(k, n)}\) とおくと, \(\zeta_n^k\) は \(1\) の原始 \(i\) 乗根.
    \(k\) が \(1\) から \(n\) まで動くとき, \(i\) は \(n\) の約数全体の値をとる.
  \end{proof}

\end{lemma}


\begin{lemma}

  最小多項式 \(m_M(x)\) のモニックな既約因子はある \(n\) の正の約数 \(i\) に対する
  第 \(i\) 円分多項式 \(\Phi_i(x)\) と一致する.

  \begin{lstlisting}
    lemma normalizedfactor_eq_cyclotomic (h : IsOfFinOrder M.val) (h' : n = orderOf M.val) (f : ℚ[X]) (fmon : Monic f) (firr : Irreducible f) (fdvd : f ∣ minpoly ℚ M.val) :
    ∃ i ∈ n.divisors, f = Φ i := by
  have npos : 0 < n := by simp only [h', orderOf_pos_iff, h]
  have hdvd : f ∣ X^n - 1 := dvd_trans fdvd (minpoly_dvd_X_pow_sub_one M h')
  rw [← prod_cyclotomic_eq_X_pow_sub_one npos] at hdvd
  have h₁ : ∃ i ∈ n.divisors, f ∣ Φ i := by
    rw [← (prime firr).dvd_finset_prod_iff]; exact hdvd
  obtain ⟨i, imem, hdvd'⟩ := h₁
  have h₂ : Irreducible (Φ i) :=
    cyclotomic.irreducible_rat (pos_of_mem_divisors imem)
  have h₃ : Associated f (Φ i) := associated_of_dvd firr h₂ hdvd'; use i, imem; rcases h₃ with ⟨u, feq⟩
  have ueq : u.val = 1 := by
    have h1 : Monic (Φ i) := cyclotomic.monic i ℚ
    rw [← feq] at h1
    have h2 : Monic u.val := Monic.of_mul_monic_left fmon h1
    rw [← Monic.isUnit_iff h2]; exact u.isUnit
  rw [← feq, ueq, mul_one]  \end{lstlisting}

  \begin{proof}
    \(f\) を \(m_M(x)\) のモニックな既約因子とする.
    補題2.2, 2.4より, \(f \mid \prod_{i \mid n} \Phi_i(x)\) が成り立つ.
    多項式環 \(\mathbb{Q}[X]\) はUFD(一意分解整域)で, \(f\) が既約であることから,
    ある \(n\) の約数 \(i\) について \(f \mid \Phi_i(x)\).
    \(f, \Phi_i(x)\) はともに既約より, ある単元 \(u \in \mathbb{Q}[X]^\times\) を用いて,
    \(\Phi_i(x) = f * u\) と書ける.
    \(f, \Phi_i(x)\) はともにモニックであるため, \(u = 1\) が示される.    
  \end{proof}

\end{lemma}


\begin{definition}[オイラーのトーシェント関数]\

  正の整数 \(n\) について, \(n\) 以下の正の整数のうち, \(n\) と互いに素であるものの個数を
  \(\phi(n)\) と書き, これをオイラーのトーシェント関数という. 
  \[\phi(n) := \# \{a \mid 1 \leq a \leq n, gcd(a, n) = 1\}.\]

\end{definition}


\begin{lemma}

  第 \(n\) 円分多項式 \(\Phi_n(x)\) の次数は, オイラーのトーシェント関数 \(\phi(n)\) に等しい.

  \begin{proof}
    円分多項式の定義から明らか.
  \end{proof}

\end{lemma}


\begin{lemma}

  \(\phi(n) \leq 2 \Leftrightarrow n \in \{1, 2, 3, 4, 6\}\).

  \begin{proof}
    \(p_j\) (\(1 \leq j \leq r\)) を相異なる素数とし, \(n = \prod_{1 \leq j \leq r} p_j^{n_j}\) と素因数分解する.
    このとき, \(\phi(n) = \phi(\prod_{1 \leq j \leq r} p_j^{n_j}) = \prod_{1 \leq j \leq r} \phi(p_j^{n_j})
    = \prod_{1 \leq i \leq n} p_i^{n_i-1}(p_i-1)\) が成り立つ.
    \(p_j \geq 5\) なる素因数が存在すれば, \(\phi(n) \geq 4\) となるため不適.
    よって, \(n\) は \(2\) と \(3\) 以外の素因数をもたない. \(n = 2^a3^b\) (\(a, b \geq 0\)) とおくと,
    \(\phi(n) = 2^a3^{b-1} \leq 2 \Leftrightarrow (a, b) = (0, 0), (1, 0), (0, 1), (2, 0), (1, 1)\)
    (\(b = 0\) の場合は, \(\phi(3^0) = 1\) とする).
    したがって, \(n = 1, 2, 3, 4, 6\).
  \end{proof}

\end{lemma}


\begin{lemma}

  最小多項式のモニックな既約因子は
  \[\Phi_1(x), \Phi_2(x), \Phi_3(x), \Phi_4(x), \Phi_6(x) \quad \text{に限る.}\]

  \begin{proof}
    補題2.1, 2.4より, 最小多項式の既約因子である円分多項式の次数は \(2\) 以下.
    補題2.5, 2.6より, \(i\) は \(1, 2, 3, 4, 6\) に限る.
  \end{proof}

\end{lemma}


\begin{lemma}
  
  最小多項式 \(m_M(x)\) は重解を持たない.

  \begin{proof}
    補題2.2より, 最小多項式は \(x^n - 1\) を割り切り, \(x^n - 1\) は複素数範囲で
    相異なる \(n\) 個の解を持つ(重解を持たない)ことから示せる.
  \end{proof}
\end{lemma}


\begin{lemma}

  \(M\) の最小多項式 \(m_M(x)\) は
  \[\Phi_1(x), \Phi_2(x), \Phi_3(x), \Phi_4(x), \Phi_6(x), \Phi_1(x)\Phi_2(x) \quad \text{に限る.}\]

  \begin{lstlisting}
    lemma minpoly_class (h : IsOfFinOrder M.val) (h' : n = orderOf M.val) : minpoly ℚ M.val ∈ ({Φ 1, Φ 2, Φ 3, Φ 4, Φ 6, (Φ 1)*(Φ 2)} : Finset ℚ[X]) := by
  obtain ⟨f, fmon, firr, fdvd⟩ := exist_normalizefactor M
  have h₀ := normalizedfactor_class M h h' f fmon firr fdvd
  rcases fdvd with ⟨g, fgeq⟩
  have h₁ := monic (isIntegral M.val)
  have gmon : Monic g := by
    rw [fgeq] at h₁; apply Monic.of_mul_monic_left fmon h₁
  have gdvd : g ∣ minpoly ℚ M.val := by
    rw [fgeq]; exact dvd_mul_left g f
  have h₃ : (minpoly ℚ M.val).natDegree = f.natDegree + g.natDegree := by rw [fgeq]; apply natDegree_mul (Monic.ne_zero fmon) (Monic.ne_zero gmon)
  have h₄ := minpoly_deg_le_two M
  have h₅ := minpoly_squarefree M h h'
  interval_cases h₆ : (minpoly ℚ M.val).natDegree
  -- (minpoly ℚ M.val).natDegree = 0
  · have := natDegree_pos (isIntegral M.val)
    rw [h₆] at this; contradiction
  -- (minpoly ℚ M.val).natDegree = 1
  · have gdeg : g.natDegree = 0 := by
      have : 1 ≤ f.natDegree := by
        simp only [mem_insert, mem_singleton] at h₀
        rcases h₀ with (rfl | rfl | rfl | rfl | rfl)
        <;> rw [cyclotomic_deg_eq_totient] <;> decide
      linarith
    have geq : g = 1 := eq_one_of_monic_natDegree_zero gmon gdeg
    rw [fgeq, geq, mul_one]
    simp only [mem_insert, mem_singleton] at h₀
    rcases h₀ with (rfl | rfl | rfl | rfl | rfl) <;> simp only [mem_insert, mem_singleton, true_or, or_true]
  -- (minpoly ℚ M.val).natDegree = 2
  · simp only [mem_insert, mem_singleton] at h₀
    rcases h₀ with (h1 | h1 | h1 | h1 | h1)
    -- f = Φ 1
    · have fdeg : f.natDegree = 1 := by rw [h1, cyclotomic_one]; compute_degree; decide
      have gdeg : g.natDegree = 1 := by rw [fdeg] at h₃; linarith
      have girr : Irreducible g := by
        rw [← degree_eq_iff_natDegree_eq_of_pos (by decide), cast_one] at gdeg
        exact irreducible_of_degree_eq_one gdeg
      simp only [mem_insert, mem_singleton]; right; right; right; right; right; rw [fgeq, h1]
      obtain h2 := normalizedfactor_class M h h' g gmon girr gdvd
      simp only [mem_insert, mem_singleton] at h2
      rcases h2 with (h3 | h3 | h3 | h3 | h3)
      -- g = Φ 1
      · rw [h3, ← h1] at fgeq
        rw [squarefree_iff_no_irreducibles (ne_zero_of_finite ℚ M.val)] at h₅
        specialize h₅ f firr; rw [fgeq] at h₅; contrapose h₅; exact dvd_refl (f * f)
      -- g = Φ 2
      · rw [h3, cyclotomic_one, cyclotomic_two]
      -- g = Φ 3
      · rw [h3] at gdeg
        have : (X^2 + X + 1 : ℚ[X]).natDegree = 2 := by compute_degree; decide
        rw [cyclotomic_three, this] at gdeg; contradiction
      -- g = Φ 4
      · rw [h3] at gdeg
        have : (X^2 + 1 : ℚ[X]).natDegree = 2 := by compute_degree; decide
        rw [cyclotomic_four, this] at gdeg; contradiction
      -- g = Φ 6
      · rw [h3] at gdeg
        have : (X^2 - X + 1 : ℚ[X]).natDegree = 2 := by compute_degree; decide
        rw [cyclotomic_six, this] at gdeg; contradiction
    -- f = Φ 2
    · have fdeg : f.natDegree = 1 := by rw [h1, cyclotomic_two]; compute_degree; decide
      have gdeg : g.natDegree = 1 := by rw [fdeg] at h₃; linarith
      have girr : Irreducible g := by rw [← degree_eq_iff_natDegree_eq_of_pos (by decide), cast_one] at gdeg; exact irreducible_of_degree_eq_one gdeg
      simp only [mem_insert, mem_singleton]; right; right; right; right; right; rw [fgeq, h1]
      obtain h2 := normalizedfactor_class M h h' g gmon girr gdvd
      simp only [mem_insert, mem_singleton] at h2
      rcases h2 with (h3 | h3 | h3 | h3 | h3)
      -- g = Φ 1
      · rw [h3, mul_comm]
      -- g = Φ 2
      · rw [h3, ← h1] at fgeq
        rw [squarefree_iff_no_irreducibles (ne_zero_of_finite ℚ M.val)] at h₅
        specialize h₅ f firr; rw [fgeq] at h₅; contrapose h₅; exact dvd_refl (f * f)
      -- g = Φ 3
      · have : (X^2 + X + 1 : ℚ[X]).natDegree = 2 := by compute_degree; decide
        rw [h3, cyclotomic_three, this] at gdeg; contradiction
      -- g = Φ 4
      · have : (X^2 + 1 : ℚ[X]).natDegree = 2 := by compute_degree; decide
        rw [h3, cyclotomic_four, this] at gdeg; contradiction
      -- g = Φ 6
      · have : (X^2 - X + 1 : ℚ[X]).natDegree = 2 := by compute_degree; decide
        rw [h3, cyclotomic_six, this] at gdeg; contradiction
    -- f = Φ 3
    · have fdeg : f.natDegree = 2 := by rw [h1, cyclotomic_three]; compute_degree; decide
      have gdeg : g.natDegree = 0 := by rw [fdeg] at h₃; linarith
      have geq : g = 1 := eq_one_of_monic_natDegree_zero gmon gdeg
      rw [fgeq, geq, mul_one, h1]; simp only [mem_insert, mem_singleton, true_or, or_true]
    -- f = Φ 4
    · have fdeg : f.natDegree = 2 := by rw [h1, cyclotomic_four]; compute_degree; decide
      have gdeg : g.natDegree = 0 := by rw [fdeg] at h₃; linarith
      have geq : g = 1 := eq_one_of_monic_natDegree_zero gmon gdeg
      rw [fgeq, geq, mul_one, h1]; simp only [mem_insert, mem_singleton, true_or, or_true]
    -- f = Φ 6
    · have fdeg : f.natDegree = 2 := by rw [h1, cyclotomic_six]; compute_degree; decide
      have gdeg : g.natDegree = 0 := by rw [fdeg] at h₃; linarith
      have geq : g = 1 := eq_one_of_monic_natDegree_zero gmon gdeg
      rw [fgeq, geq, mul_one, h1]; simp only [mem_insert, mem_singleton, true_or, or_true]  \end{lstlisting}

  \begin{proof}
    補題2.7より, 最小多項式 \(m_M(x)\) のモニックな既約因子は
    \(\Phi_1(x), \Phi_2(x), \Phi_3(x), \Phi_4(x), \Phi_6(x)\)
    に限られ, 補題2.8より, 
    同一の円分多項式は重複して現れない.
    よって補題2.1から, 考えられる最小多項式は \(\Phi_1(x), \Phi_2(x), \Phi_3(x), \Phi_4(x), \Phi_6(x), \Phi_1(x)\Phi_2(x)\)
    に限られる.
  \end{proof}

\end{lemma}


\begin{theorem}

  \(GL(2, \mathbb{Q})\) の有限位数の元の位数は \(1, 2, 3, 4, 6\) に限る.

  \begin{proof}
    \(m_M(x) = \Phi_i(x)\) (\(i = 1, 2, 3, 4, 6\)) のとき,
    \(m_M(M) = O\) より, \(\Phi_i(M) = O\).
    \(\Phi_i(x) \mid x^i - 1\) から, \(M^i = I\).
    また \(j (< i)\) について, \(\Phi_i(x) \nmid x^j - 1\) より, \(M^j \neq I\).
  
    \(m_M(x) = \Phi_1(x)\Phi_2(x)\) のとき, \(m_M(x) = x^2 - 1\) より, \(M^2 = I\).
    また, \(m_M(x) \neq x-1\) より, \((x - 1)(M) \neq O\).
  \end{proof}

\end{theorem}

\begin{theorem}
  
  \(n = 1, 2, 3, 4, 6\) それぞれに対して, 位数が \(n\) となるような
  \(GL(2, \mathbb{Q})\) の元 \(M\) が存在する.

  \begin{proof}
    \(\left(\begin{smallmatrix} 1 & 0 \\ 0 & 1 \end{smallmatrix} \right)\) は位数1の元,
    \(\left(\begin{smallmatrix} 1 & 0 \\ 0 & -1 \end{smallmatrix} \right)\) は位数2の元,
    \(\left(\begin{smallmatrix} 0 & -1 \\ 1 & -1 \end{smallmatrix} \right)\) は位数3の元,
    \(\left(\begin{smallmatrix} 0 & 1 \\ -1 & 0 \end{smallmatrix} \right)\) は位数4の元,
    \(\left(\begin{smallmatrix} 0 & -1 \\ 1 & 1 \end{smallmatrix} \right)\) は位数6の元であり,
    これらの行列式は \{\(-1, 1\)\}.
  \end{proof}

\end{theorem}

\section{おわりに}

  以上で形式化を終えたわけだが, 掲載したコードは完成形のほんの一部である.
  形式化を困難にしている要因は, 一見自明に思える事実の証明や場合分けの多さにある.
  すべてのパターンを抜け目なく網羅できる点は, Leanの利点でもあり, 大変な部分でもある.
  今回は有理数体上で考えたが, 実数体や複素数体へ拡張すると, 代数的構造はより複雑になり,
  さらなる困難が予想される.

\end{multicols}

\end{document}